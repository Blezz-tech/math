
\begin{tcolorbox}[
    colback=blue!5!white,
    colframe=blue!75!black,
    title=Задание 1.1
  ]

  $15$-го марта планируется взять кредит в банке на 26 месяцев. Условия его возврата таковы:

  \begin{enumerate}[--]
    \item $1$-го числа каждого месяца долг возрастает на $3$\% по сравнению с концом предыдущего месяца;
    \item со $2$-го по $14$-е число каждого месяца необходимо выплатить часть долга;
    \item $15$-го числа каждого месяца с $1$-го по $25$-й долг должен быть на $40$ тысяч рублей меньше долга на $15$-е число предыдущего месяца;
    \item к $15$-му числу $26$-го месяца кредит должен быть полностью погашен.
  \end{enumerate}

  Какую сумму планируется взять в кредит, если общая сумма выплат после полного его погашения составит $1924$ тысячи рублей?

\end{tcolorbox}

\begin{tcolorbox}[
    colback=white!100!white,
    colframe=green!75!black,
    title=Решение 1.1
  ]


  Пусть

  \begin{equation*}
    \begin{array}{ll}
      S - ?      & \text{- сумма кредита}         \\
      S_n = 1924 & \text{- итоговая сумма выалат} \\
      r = 0,03   & \text{- кредитная ставка}      \\
    \end{array}
  \end{equation*}

  Составим таблицу выплат

  \begin{tabular}{|l|l|l|l|}
    \hline
    $1$     & $r\cdot S$                & $r\cdot S + 40$                & $S - 40$        \\ \hline
    $2$     & $r\cdot (S - 40)$         & $r\cdot (S - 40) + 40$         & $S - 2\cdot40$  \\ \hline
    $3$     & $r\cdot (S - 2\cdot 40)$  & $r\cdot (S - 2\cdot 40) + 40$  & $S - 3\cdot40$  \\ \hline
    $\dots$ & $\dots$                   & $\dots$                        & $\dots$         \\ \hline
    $24$    & $r\cdot (S - 23\cdot 40)$ & $r\cdot (S - 23\cdot 40) + 40$ & $S - 24\cdot40$ \\ \hline
    $25$    & $r\cdot (S - 24\cdot 40)$ & $r\cdot (S - 24\cdot 40) + 40$ & $S - 25\cdot40$ \\ \hline
    $26$    & $r\cdot (S - 25\cdot 40)$ & $r\cdot (S - 25\cdot 40) + 40$ & $0$             \\ \hline
  \end{tabular}

  $S_n = r\cdot(S + S - 40 + S - 2\cdot 40 \dots S - 23\cdot 40 + S - 24\cdot 40 + S - 25\cdot 40) + 40\cdot 26 = 1924$

  Мы видим арифметическую прогрессию:

  $F = S + S - 40 + S - 2\cdot 40 \dots S - 23\cdot 40 + S - 24\cdot 40 + S - 25\cdot 40$

  Найдём сумму арифмитической прогрессии:

  \begin{equation*}
    \begin{array}{l}
      F=\frac{a_1+a_n}{2}\cdot n             \\
      F=\frac{S + S - 25\cdot 40}{2}\cdot 26 \\
      F=(2S - 1000)\cdot 13                  \\
      F= 26S - 13000                         \\
    \end{array}
  \end{equation*}

  Итого получаем:

  \begin{equation*}
    \begin{array}{l}
      S_n = r(26S - 13000) + 25\cdot 40 + (S - 25\cdot 40) = 1924 \\
      0,03 (26S - 13000) + S = 1924                               \\
      3 (26S - 13000) + 100S = 192400                             \\
      78S - 39000 + 100S = 192400                                 \\
      178S = 231400                                               \\
      S = 178
    \end{array}
  \end{equation*}

  Ответ: $178$

\end{tcolorbox}

