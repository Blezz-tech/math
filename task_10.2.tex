\section*{Задание 10.2}

Смешав 45\%-ный и 97\%-ный растворы кислоты и добавив 10 кг чистой воды, получили 62\%-ный раствор кислоты. Если бы вместо 10 кг воды добавили 10 кг 50\%-ного раствора той же кислоты, то получили бы 72\%-ный раствор кислоты. Сколько килограммов 45\%-ного раствора использовали для получения смеси?

\begin{equation*}
    \begin{array}{rcl}
        k_1m_1 + k_2m_2 + \ldots & = & k_3m_3 \\
        m_1 + m_2 + \ldots       & = & m_3
    \end{array}
\end{equation*}

\begin{equation*}
    \left\{
    \begin{array}{lclr}
        0,45\cdot m_1 + 0,97\cdot m_2 + 0 \cdot 10  & = & 0,62(m_1 + m_2 + 10) & (1) \\
        0,45\cdot m_1 + 0,97\cdot m_2 + 0,5\cdot 10 & = & 0,72(m_1 + m_2 + 10) & (2)
    \end{array}
    \right.
\end{equation*}

\begin{equation*}
    \begin{array}{rcl}
        0,1 (m_1+m_2+10) & = & 5        \\
        10 + m_1 + m_2   & = & 50       \\
        m_1 + m_2        & = & 40       \\
        m_2              & = & 40 - m_1
    \end{array}
\end{equation*}

\begin{equation*}
    \begin{array}{rcl}
        0,45\cdot m_1 + 0,97(40-m_1)   & = & 0,62\cdot 50 \\
        0,45\cdot m_1 + 38,8 - 0,97m_1 & = & 31           \\
        7,8                            & = & 0,52m_1      \\
        52m_1                          & = & 780          \\
        m_1                            & = & 15
    \end{array}
\end{equation*}

Ответ: $15$

\subsection*{Подзадание 50}

Смешав 43-процентный и 89-процентный растворы кислоты и добавив 10 кг чистой воды, получили 69-процентный раствор кислоты. Если бы вместо 10 кг воды добавили 10 кг 50-процентного раствора той же кислоты, то получили бы 73-процентный раствор кислоты. Сколько килограммов 43-процентного раствора использовали для получения смеси?

\begin{equation*}
    \left\{
    \begin{array}{lclr}
        0,43 \cdot m_1 + 0,89 \cdot m_2 + 0 \cdot 10   & = & 0,69 (m_1 + m_2 + 10) & (1) \\
        0,43 \cdot m_1 + 0,89 \cdot m_2 + 0,5 \cdot 10 & = & 0,73 (m_1 + m_2 + 10) & (2)
    \end{array}
    \right.
\end{equation*}

\begin{equation*}
    \begin{array}{rcl}
        0,04(m_1 + m_2 + 10) & = & 5         \\
        m_1 + m_2 + 10       & = & 125       \\
        m_2                  & = & 115 - m_1
    \end{array}
\end{equation*}

\begin{equation*}
    \begin{array}{rcl}
        0,43 \cdot m_1  + 0,89 (115 - m_1) & = & 0,69 (m_1 + 115 - m_1 + 10)           \\
        0,43 \cdot m_1 - 0,89 \cdot m_1    & = & 0,69 \cdot 115 + 6,9 - 0,89 \cdot 115 \\
        0,46 \cdot m_1                     & = & 0,2 \cdot 115 - 6,9                   \\
        0,46 \cdot m_1                     & = & 23 - 6,9                              \\
        0,46 \cdot m_1                     & = & 16,1                                  \\
        m_1                                & = & \frac{1610}{46}                       \\
        m_1                                & = & \frac{805}{23}                        \\
        m_1                                & = & 35                                    \\
    \end{array}
\end{equation*}

\subsection*{Подзадание 51}

51. Смешав 38-процентный и 52-процентный растворы кислоты и добавив 10 кг чистой воды, получили 36-процентный раствор кислоты. Если бы вместо 10 кг воды добавили 10 кг 50-процентного раствора той же кислоты, то получили бы 46-процентный раствор кислоты. Сколько килограммов 38-процентного раствора использовали для получения смеси?

\begin{equation*}
    \left\{
    \begin{array}{lclr}
        0,38 \cdot m_1 + 0,52 \cdot m_2 + 0 \cdot 10   & = & 0,36 (m_1 + m_2 + 10) & (1) \\
        0,38 \cdot m_1 + 0,52 \cdot m_2 + 0,5 \cdot 10 & = & 0,46 (m_1 + m_2 + 10) & (2)
    \end{array}
    \right.
\end{equation*}

\begin{equation*}
    \begin{array}{rcl}
        0,1(m_1 + m_2 + 10) & = & 5        \\
        m_1 + m_2 + 10      & = & 50       \\
        m_2                 & = & 40 - m_1
    \end{array}
\end{equation*}

\begin{equation*}
    \begin{array}{rcl}
        0,38 \cdot m_1 + 0,52(40 - m_1) & = & 0,36(m _1 + 40 - m_1 + 10)           \\
        0,38 \cdot m_1 - 0,52 \cdot m_1 & = & 3,6 + 0,36 \cdot 40  - 0,52 \cdot 40 \\
        0,52 \cdot m_1 - 0,38 \cdot m_1 & = & 0,16 \cdot 40 - 4,6                  \\
        0,14 \cdot m_1                  & = & 6,4 - 4,6                            \\
        0,14 \cdot m_1                  & = & 2,8                                  \\
        m_1                             & = & \frac{280}{14}                       \\
        m_1                             & = & 20                                   \\
    \end{array}
\end{equation*}

