
\begin{tcolorbox}[
    colback=blue!5!white,
    colframe=blue!75!black,
    title=Задание 3.1
  ]

  B первом цилиндрическом сосуде уровень жидкости достигает $16$ см. Эту жидкость перелили во второй цилиндрический сосуд, диаметр основания которого в $2$ раза больше диаметра основания первого. На какой высоте будет находиться уровень жидкости во втором сосуде? Ответ дайте в сантиметрах.

\end{tcolorbox}

\begin{tcolorbox}[
    colback=white!100!white,
    colframe=green!75!black,
    title=Решение 3.1
  ]

  \begin{center}
    \begin{asy}
      import solids;
      size(0,150);
      currentlight=Viewport;

      triple start = (0.5,0.5,0);
      real h = 1;
      real r = 0.5;
      triple az = (0,0,1);
      revolution cyl = cylinder(start,r,h,az);
      draw(cyl,0.5+black);


      triple start1 = (-0.32,1.75,0);
      real h1 = 1;
      real r1 = 1;
      triple az1 = (0,0,1);
      revolution cyl1 = cylinder(start1,r1,h1,az1);
      draw(cyl1,0.5+black);

      triple start2 = (0.5,0.5,0);
      real h2 = 0.5;
      real r2 = 0.5;
      triple az2 = (0,0,1);
      revolution cyl2 = cylinder(start2,r2,h2,az2);
      draw(cyl2,0.7+blue);


      triple start3 = (-0.32,1.75,0);
      real h3 = 0.2;
      real r3 = 1;
      triple az3 = (0,0,1);
      revolution cyl3 = cylinder(start3,r3,h3,az3);
      draw(cyl3,0.7+blue);
    \end{asy}
  \end{center}
  \vspace{2mm}

  \begin{equation*}
    \begin{array}{l}
      V = S_\text{Осн} \cdot h = \pi R^2 \cdot 16 = \pi (2R)^2 \cdot h \\
      R^2 \cdot 16 = 4 R^2 h                                           \\
      h=4                                                              \\
    \end{array}
  \end{equation*}

  Ответ: $4$

\end{tcolorbox}

