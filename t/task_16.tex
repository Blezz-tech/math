\section*{Задание 15.1}

% В июле 2025 года планируется взять кредит на десять лет в размере 800 тыс. рублей. Условия его возврата таковы:

% \begin{enumerate}[--]
%     \item каждый январь долг будет возрастать на $r\%$ по сравнению с концом предыдущего года ($r$ - целое число);
%     \item с февраля по июнь каждого года необходимо оплатить одним платежом часть долга;
%     \item в июле 2026, 2027, 2028, 2029 и 2030 годов долг должен быть на какую-то одну и ту же величину меньше долга на июль предыдущего года;
%     \item в июле 2030 года долг должен составить 200 тыс. рублей;
%     \item в июле 2031, 2032, 2033, 2034 и 2035 годов долг должен быть на другую одну и ту же величину меньше долга на июль предыдущего года;
%     \item к июлю 2035 года долг должен быть выплачен полностью.
% \end{enumerate}

% Известно, что сумма всех платежей после полного погашения кредита будет равна 1480 тыс. рублей. Найдите r.


Пусть

\begin{itemize}
    \item $S = 800$ тыс. руб. кредит
    \item $x$ - Выплата с 26-30 гг.
    \item $y$ - Выплата с 31-35 гг.
    \item $k = \frac{r}{100}$ - Процентная ставка
\end{itemize}

Составим таблицу выплат


\begin{tabular}{|l|l|l|l|}
    \hline
    $26$ & $k\cdot S$          & $k\cdot S + x$          & $S - x$        \\ \hline
    $27$ & $k\cdot (S - x)$    & $k\cdot (S - x) + x$    & $S - 2x$       \\ \hline
    $27$ & $k\cdot (S - 2x)$   & $k\cdot (S - 2x) + x$   & $S - 3x$       \\ \hline
    $27$ & $k\cdot (S - 3x)$   & $k\cdot (S - 3x) + x$   & $S - 4x$       \\ \hline
    $27$ & $k\cdot (S - 4x)$   & $k\cdot (S -4x) + x$    & $S - 5x = 200$ \\ \hline
    $31$ & $k\cdot 200$        & $k\cdot 200 + y$        & $200 - y$      \\ \hline
    $32$ & $k\cdot (200 - y)$  & $k\cdot (200 - y) + y$  & $200 - 2y$     \\ \hline
    $33$ & $k\cdot (200 - 2y)$ & $k\cdot (200 - 2y) + y$ & $200 - 3y$     \\ \hline
    $34$ & $k\cdot (200 - 3y)$ & $k\cdot (200 - 3y) + y$ & $200 - 4y$     \\ \hline
    $35$ & $k\cdot (200 - 4y)$ & $k\cdot (200 - 4y) + y$ & $200 - 5y = 0$ \\ \hline
\end{tabular}

Необходимые коэффиценты

$S - 5x = 200 \Rightarrow 5x = S - 200 \Rightarrow 5x = 800 - 200 \Rightarrow 5x = 600 \Rightarrow x = 120$

$200 - 5y = 0 \Rightarrow 5y = 200 \Rightarrow y = 40$

Найдём выплаты за первые 5 лет

\begin{enumerate}
    \item $5x + k(S + S - x + S - 2x + S - 3x + S - 4x)$
    \item $5x + k(5S - 10x)$
    \item $5x + 5kS - 10kx$
    \item $5\cdot 120 + 5k\cdot 800 - 10k\cdot 120$
    \item $600 + 4000k - 1200k$
    \item $2800k + 600$
\end{enumerate}

Найдём выплаты за вторые пять лет

\begin{enumerate}
    \item $5y + k(200 + 200 - y + 200 - 2y + 200 - 3y + 200 - 4y)$
    \item $5y + k(1000 - 10y)$
    \item $5y + 1000k - 10ky$
    \item $5\cdot 40 + 1000k - 10k \cdot 40$
    \item $200 + 1000k - 400k$
    \item $600k + 200$
\end{enumerate}

Выплат за 10 лет

\begin{equation*}
    \begin{array}{rcl}
        (2800k + 600) + (600k + 200) & = & 1480            \\
        3400k                        & = & 1480 - 800      \\
        \frac{3400r}{100}            & = & 680            \\
        17r                          & = & 340            \\
        r                            & = & 20 \\
    \end{array}
\end{equation*}

Ответ: 20